\documentclass[12pt,letterpaper]{hmcpset}
\usepackage[margin=1in]{geometry} 
\usepackage{graphicx}
\usepackage{amsmath}
\usepackage{mathrsfs}

% info for header block in upper right hand corner
\name{ }
\class{Math 65}
\assignment{HW 2}
\duedate{Wednesday, May 18, 2016}

\newcommand{\pn}[1]{\left( #1 \right)}
\newcommand{\abs}[1]{\left| #1 \right|}
\newcommand{\bk}[1]{\left[ #1 \right]}

\newcommand{\RR}{\mathbb{R}}
\newcommand{\m}[1]{\begin{bmatrix} #1 \end{bmatrix}}

\renewcommand{\labelenumi}{{(\alph{enumi})}}

\begin{document}

\problemlist{6.2.\{20, 25, 28, 38\}, 6.3.\{7, 18, 21\}, Additional Problem \#1}

\begin{problem}[6.2.20]
    In Exercises 18-25, determine whether the set $\mathcal{B}$ is a
    basis for the vector space $V$.\\\\
    $V=M_{22},\mathcal{B}=\left\{\m{1&0\\0&1},\m{0&1\\1&0},\m{1&1\\0&1},
    \m{1&0\\1&1}\right\}$
\end{problem}
\begin{solution}
    \vfill
\end{solution}
\newpage

\begin{problem}[6.2.25]
    In Exercises 18-25, determine whether the set $\mathcal{B}$ is a
    basis for the vector space $V$.\\\\
    $V=\mathscr{P}_2,\mathcal{B}=\{1,2-x,3-x^2,x+2x^2\}$
\end{problem}
\begin{solution}
    \vfill
\end{solution}
\newpage

\begin{problem}[6.2.28]
    Find the coordinate vector of $p(x)=1+2x+3x^2$ with respect to the basis
    $\mathcal{B}=\{1,1+x,-1+x^2\}$ of $\mathscr{P}_2$.
\end{problem}
\begin{solution}
    \vfill
\end{solution}
\newpage

\begin{problem}[6.2.38]
    In Exercises 34-39, find the dimension of the vector space $V$ and give a
    basis for $V$.\\\\\
    $V=\{A$ in $M_{22}:A$ is skew-symmetric\}
\end{problem}
\begin{solution}
    \vfill
\end{solution}
\newpage

\begin{problem}[6.3.7]
    In Exercises 5-8, follow the instructions for Exercises 1-4 using $p(x)$
    instead of \textbf{x}.\\\\
    $p(x)=1+x^2,\mathcal{B}=\{1+x+x^2,x+x^2,x^2\},\mathcal{C}=\{1,x,x^2\}$ in
    $\mathscr{P}_2$
\end{problem}
\begin{solution}
    \vfill
\end{solution}
\newpage

\begin{problem}[6.3.18]
    Express $p(x)=1+2x-5x^2$ as a Taylor polynomial about $a=-2$.
\end{problem}
\begin{solution}
    \vfill
\end{solution}
\newpage

\begin{problem}[6.3.21]
    Let $\mathcal{B},\mathcal{C}$, and $\mathcal{D}$ be bases for a
    finite-dimensional vector space $V$. Prove that
    $$P_{\mathcal{D}\leftarrow\mathcal{C}}P_{\mathcal{C}\leftarrow\mathcal{B}}
    =P_{\mathcal{D}\leftarrow\mathcal{B}}$$
\end{problem}
\begin{solution}
    \vfill
\end{solution}
\newpage

\begin{problem}[Additional Problem \#1]
    For each square matrix below, calculate its eigenvalues and eigenvectors.
    Then verify that $PDP^{-1}$ is equal to the original matrix, where $D$ is a
    diagonal matrix with your eigenvalues along its diagonal and $P$ is a matrix
    with your eigenvectors as its columns.
    \begin{enumerate}
        \item $\m{1&-0\\0&0}$
        \item $\m{0&-13&-4\\0&-3&0\\1&13&0}$
    \end{enumerate}
\end{problem}
\begin{solution}
    \vfill
\end{solution}
\end{document}