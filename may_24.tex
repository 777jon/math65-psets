\documentclass[12pt,letterpaper]{hmcpset}
\usepackage[margin=1in]{geometry}
\usepackage{graphicx}
\usepackage{amsmath}
\usepackage{mathrsfs}

% info for header block in upper right hand corner
\name{ }
\class{Math 65}
\assignment{HW 6}
\duedate{Tuesday, May 24, 2016}

\newcommand{\pn}[1]{\left( #1 \right)}
\newcommand{\abs}[1]{\left| #1 \right|}
\newcommand{\bk}[1]{\left[ #1 \right]}

\newcommand{\RR}{\mathbb{R}}
\newcommand{\m}[1]{\begin{bmatrix} #1 \end{bmatrix}}
\newcommand{\hatbf}[1]{\hat{\mathbf{#1}}}

\renewcommand{\labelenumi}{{(\alph{enumi})}}

\begin{document}

\problemlist{1, 2, 3, 4, 5}

\noindent
\emph{Intructor's Note:} You need to be able to calculate eigenvalues
and eigenvectors by hand on the final exam. If you haven't honed this
skill well enough to the point where you can do it quickly and
error-free, then perform the calculations on your homework assignments
by hand. If you don't need the practice, then feel free to use the
computer to do those calculations for you.

\begin{problem}[1]
    \[
        \text{Let }A=\m{2&0&0&1\\0&1&0&0\\0&0&1&0\\1&0&0&2}.
    \]
    \begin{enumerate}
        \item Find eigenvalues and eigenvectors of this symmetric
            matrix. (Review Poole Section 4.2 if you've forgotten how to
            calculate a determinant using cofactor expansions.)\\
            \textbf{Note:} Eigenvectors corresponding to the same
            eigenvalue don't have to be orthogonal, but you can choose
            them so that all four eigenvectors will form an orthogonal
            basis for $\RR^4$.
        \item Normalize all of your eigenvectors (that is, scale them
            so they have unit length) and arrange them as columns of a
            matrix $Q$. Then verify that $Q^TQ=I_4$ and $QQ^T=I_4$.
        \item Calculate $Q^TAQ$. (Before you calculate it, think about
            what you expect it to be.)
        \item Let $\hatbf{v}_1,\dots,\hatbf{v}_4$ be your orthogonal
            eigenvectors. Since they are linearly independent, they form
            an orthogonal basis for $\RR^4$. Let
            $B=\{\hatbf{v}_1,\dots,\hatbf{v}_4\}$. \textit{Without solving
            a simultaneous system of four coupled equations for four
            unknowns}, calculate
            \[
                \m{1\\2\\3\\4}_B
            \]
    \end{enumerate}
\end{problem}
\newpage
\begin{solution}
    \null\vfill
\end{solution}
\newpage

\begin{problem}[2]
    Let $W$ be the subspace of $\RR^4$ that is spanned by these three
    vectors:
    \[
        \mathbf{x}_1=\m{2\\-1\\1\\2},\qquad\mathbf{x}_2=\m{3\\-1\\0\\4},\qquad
        \mathbf{x}_3=\m{1\\1\\1\\1}.
    \]
    \begin{enumerate}
        \item Use the Gram-Schmidt Process to calculate $\mathbf{v}_1,
            \mathbf{v}_2,\mathbf{v}_3$, three vectors that form an
            orthogonal basis for $W$.
        \item Normalize $\mathbf{v}_1,\mathbf{v}_2,\mathbf{v}_3$ to
            get $\{\hatbf{v}_1,\hatbf{v}_2,\hatbf{v}_3\}$, an
            orthonormal basis for $W$.
        \item Find a unit vector $\hatbf{v}_4$ that allows
            $B=\{\hatbf{v}_1,\dots,\hatbf{v}_4\}$ to be an orthonormal
            basis for $\RR^4$.\\
            \textbf{Hint:} One way to do this to construct a matrix
            matrix $Q$ whose columns are
            $\hatbf{v}_1,\dots,\hatbf{v}_4$. Find the missing fourth
            column of the matrix so that $QQ^T=Q^TQ=I$.
        \item \textit{Without solving a simultaneous system of four
            coupled equations for four unknowns}, calculate
            \[
                \m{1\\2\\3\\4}_B
            \]
    \end{enumerate}
\end{problem}
\begin{solution}
    \vfill
\end{solution}
\newpage

\begin{problem}[3]
    Follow the method outlined in the lecture notes from Friday, May
    20 to find the general solution to
    \[
        \begin{cases}
            x_1'(t)=-x_1+2x_2\\
            x_2'(t)=2x_1-4x_2.
        \end{cases}
    \]
    Then, find the specific solution corresponding to the initial
    conditions $x_1(0)=1,x_2(0)=1$, and to describe what happens to
    this solution as $t\to\infty$.\\\\
    \textbf{Note:} Follow the method in the lecture notes
    closely. Make sure you understand each step. In subsequent
    problems, you can skip any steps in the derivation that you have
    already explained.
\end{problem}
\begin{solution}
    \vfill
\end{solution}
\newpage

\begin{problem}[4]
    Find the general solution to
    \[
        \begin{cases}
            x_1'(t)=x_1/2+9x_2\\
            x_2'(t)=x_1/2+2x_2.
        \end{cases}
    \]
    Then, find the specific solution corresponding to the initial
    conditions $x_1(0)=1,x_2(0)=1$, and to describe what happens to
    this solution as $t\to\infty$.
\end{problem}
\begin{solution}
    \vfill
\end{solution}
\newpage

\begin{problem}[5]
    Find the general solution to
    \[
        \begin{cases}
            x_1'(t)=-2x_1-x_2\\
            x_2'(t)=x_1-4x_2.
        \end{cases}
    \]
    Then, find the specific solution corresponding to the initial
    conditions $x_1(0)=1,x_2(0)=1$, and to describe what happens to
    this solution as $t\to\infty$.\\\\
    \textbf{Hint:} This time, you will find that the matrix cannot be
    diagonalized. However, the eigenvalue $\lambda$ that you obtain is
    still useful. Let $x_1(t)=u_1(t)e^{\lambda t}$ and
    $x_2(t)=u_2(t)e^{\lambda t}$ and plug these into the two DEs to
    get two new DEs for $u_1(t)$ and $u_2(t)$ that are easier to
    solve.
\end{problem}
\begin{solution}
    \vfill
\end{solution}
\end{document}
