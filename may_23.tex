\documentclass[12pt,letterpaper]{hmcpset}
\usepackage[margin=1in]{geometry}
\usepackage{graphicx}
\usepackage{amsmath}
\usepackage{mathrsfs}

% info for header block in upper right hand corner
\name{ }
\class{Math 65}
\assignment{HW 5}
\duedate{Monday, May 23, 2016}

\newcommand{\pn}[1]{\left( #1 \right)}
\newcommand{\abs}[1]{\left| #1 \right|}
\newcommand{\bk}[1]{\left[ #1 \right]}

\newcommand{\RR}{\mathbb{R}}
\newcommand{\m}[1]{\begin{bmatrix} #1 \end{bmatrix}}

\renewcommand{\labelenumi}{{(\alph{enumi})}}

\begin{document}

\problemlist{6.1.47, 6.3.16, 6.4.\{4, 21\}, 6.5.12, 6.6.2, 4.4.22, EC: 6.2.33}

\noindent
\emph{Intructor's Note:} I recommend that you also look at the Chapter Review on
pages 527-528 of Poole and skim the problems to see if there are any concepts or
problems that seem challenging to you. Try some of these problems for more
practice.

\begin{problem}[Poole 6.1.47]
    Let $V$ be a vector space with subspaces $U$ and $W$. Give an
    example with $V=\RR^2$ to show that $U\cup W$ need not be a
    subspace of $V$.
\end{problem}
\begin{solution}
    \vfill
\end{solution}
\newpage

\begin{problem}[Poole 6.3.16]
    Let $\mathcal{B}$ and $\mathcal{C}$ be bases for
    $\mathscr{P}_2$. If $\mathcal{B}=\{x,1+x,1-x+x^2\}$ and the
    change-of-basis matrix from $\mathcal{B}$ to $\mathcal{C}$ is
    \[
        P_{\mathcal{C}\leftarrow\mathcal{B}}=\m{1&0&0\\0&2&1\\-1&1&1}
    \]
    find $\mathcal{C}$.
\end{problem}
\begin{solution}
    \vfill
\end{solution}
\newpage

\begin{problem}[Poole 6.4.4]
    In Exercises 1-12, determine whether $T$ is a linear transformation.
    \[
        T:M_{nn}\to M_{nn}\text{ defined by }T(A)=AB-BA\text{, where }B
        \text{ is a fixed }n\times n\text{ matrix}
    \]
\end{problem}
\begin{solution}
    \vfill
\end{solution}
\newpage

\begin{problem}[Poole 6.4.21]
    Prove Theorem 6.14(b).
\end{problem}
\begin{solution}
    \vfill
\end{solution}
\newpage

\begin{problem}[Poole 6.5.12]
    In Exercises 9-14, find either the nullity or the rank of $T$ and
    then use the Rank Theorem to find the other.
    \[
        T:M_{22}\to M_{22}\text{ defined by }T(A)=AB-BA\text{, where }
        B=\m{1&1\\0&1}
    \]
\end{problem}
\begin{solution}
    \vfill
\end{solution}
\newpage

\begin{problem}[Poole 6.6.2]
    In Exercises 1-12, find the matrix
    $[T]_{\mathcal{C}\leftarrow\mathcal{B}}$ of the linear
    transformation $T:V\to W$ with respect to the bases $\mathcal{B}$
    and $\mathcal{C}$ of $V$ and $W$, respectively. Verify Theorem
    6.26 for the vector \textbf{v} by computing $T(\textbf{v})$
    directly and using the theorem.
    \begin{align*}
        T&:\mathscr{P}_1\to\mathscr{P}_1\text{ defined by}\\T&(a+bx)=b-ax,\\
        \mathcal{B}&=\{1+x,1-x\},\\\mathcal{C}&=\{1,x\},\\\textbf{v}&=p(x)=4+2x
    \end{align*}
\end{problem}
\begin{solution}
    \vfill
\end{solution}
\newpage

\begin{problem}[Poole 4.4.22]
    In Exercises 16-23, use the method of Example 4.29 to compute the
    indicated power of the matrix.
    \[
        \m{2&0&1\\1&1&1\\1&0&2}^k
    \]
    (assume that $k$ is a positive integer)
\end{problem}
\begin{solution}
    \vfill
\end{solution}
\newpage

\begin{problem}[Extra Credit: Poole 6.2.33]
    Let $\{\textbf{u}_1,...,\textbf{u}_m\}$ be a set of vectors in an
    $n$-dimensional vector space $V$ and let $\mathcal{B}$ be a basis
    for $V$. Let
    $S=\{[\textbf{u}_1]_{\mathcal{B}},...,[\textbf{u}_m]_{\mathcal{B}}\}$
    be on the set of coordinate vectors of
    $\{\textbf{u}_1,...,\textbf{u}_m\}$ with respect to
    $\mathcal{B}$. Prove that span$(\textbf{u}_1,...,\textbf{u}_m)=V$
    if and only if span$(S)=\RR^n$.\\\\
    (Remember that to prove an if-and-only-if theorem, you need to
    prove both directions.)
\end{problem}
\begin{solution}
    \vfill
\end{solution}
\end{document}
