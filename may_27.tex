\documentclass[12pt,letterpaper]{hmcpset}
\usepackage[margin=1in]{geometry}
\usepackage{graphicx}

% info for header block in upper right hand corner
\name{ }
\class{Math 65}
\assignment{HW 9}
\duedate{Friday, May 27, 2016}

\newcommand{\pn}[1]{\left( #1 \right)}
\newcommand{\abs}[1]{\left| #1 \right|}
\newcommand{\bk}[1]{\left[ #1 \right]}
\newcommand{\m}[1]{\begin{bmatrix} #1 \end{bmatrix}}
\newcommand{\f}[2]{\frac{#1}{#2}}
\renewcommand{\bf}[1]{\mathbf{#1}}
\renewcommand{\labelenumi}{{(\alph{enumi})}}

\begin{document}

\problemlist{1, 2, 3}

\begin{problem}[1]
    Consider the inhomogeneous linear system of differential equations
    \[
        \bf{x}'=\m{5&-4\\6&-5}\bf{x}+\m{e^{t}\\0}.
    \]
    Note that the system matrix has eigenvalues $\lambda_1=1$ and
    $\lambda_2=-1$ with corresponding eigenvectors $\bf{v}_1=\m{1\\1}$
    and $\bf{v}_2=\m{2\\3}$.
    \begin{enumerate}
        \item Assume that the solution to this system $\bf{x}$ has the
            form
            $\m{a_1e^{-t}+b_1e^{t}+c_1te^{t}\\a_2e^{-t}+b_2e^{t}+c_2te^{t}}$
            for constants $a,b,c,d,e\mbox{ and }f$.  Use the method of
            undetermined coefficients to find the general solution to
            this system.
        \item Why was the assumption made in part (a) reasonable given
            what you know about the matrix $A$?
        \item Recalculate the solution using the integrating factor
            formula.  Reconcile your answer from this calculation with
            your previous calculation to show they are equivalent.
    \end{enumerate}
\end{problem}
\begin{solution}
    \vfill
\end{solution}
\newpage

\begin{problem}[2]
    Solve the initial value problem
    \[
        \bf{x}'(t)=\m{0&2\\-2&0}\bf{x}(t)
        +\m{\cos(\omega t)\\-\sin(\omega t)}
    \]
    where $\omega$ is a nonzero constant and
    $\bf{x}(0)=\m{a\\b}$. Make sure your solution is valid for all
    values of $\omega\neq0$.  In particular, be careful not to divide
    by 0 when $\omega=-2$.  Work out the solution in the case when
    $\omega\neq-2$, then work out the solution in the case when
    $\omega=-2$.  You may use either the undetermined coefficients or
    integrating factor method.  You will get extra credit if you
    perform both calculations and show that they are equivalent.\\\\
    \textbf{Hint:} If you're using the undetermined coefficients
    method, think about what we did in Math 45 when the forcing
    function took on the same form as the homogeneous solution.
\end{problem}
\begin{solution}
    \vfill
\end{solution}
\newpage

\begin{problem}[3]
    As we will discuss in class on Friday, closed-form solutions to
    non-constant-coefficient linear systems of differential equations
    are generally unavailable.  However, if you are lucky enough to
    solve a homogeneous (unforced) non-constant-coefficient linear
    system, solving the inhomogeneous (forced) version is relatively
    easy.  In this exercise, we'll walk you through how this works.\\\\
    Consider the following initial value problem:
    \[
        \bf{x}'(t)=A(t)\bf{x}+\bf{F}(t)=\m{2&-2e^{-t}\\e^{t}&0}\bf{x}
        +\m{4\\6}\qquad\text{with}\qquad\bf{x}(0)=\m{-1\\1}.
    \]
    \begin{enumerate}
        \item First, verify that
            \[
                \bf{x}_h(t)=c_1\m{2e^{t}\\e^{2t}}+c_2\m{1\\e^{t}}
            \]
            is the general solution to the homogeneous version of the ODE
            (that is, the version of the ODE without the forcing term
            $\bf{F}$) for any $c_1$ and $c_2$.
        \item Arrange the two linearly independent solutions from part
            (a) as columns in a matrix $\Psi(t)$, which is called a
            \textit{fundamental matrix}. Verify that
            $\Psi'(t)=A(t)\Psi(t)$.
        \item The general solution to homogeneous equation is
            $\bf{x}_h(t)=\Psi(t)\bf{c}$ for some constant vector
            $\bf{c}=\m{c_1\\c_2}$. What must $\bf{c}$ be to satisfy the
            initial condition $\bf{x}_h(t_0)=\bf{x}_0$?
        \item The \textit{transition matrix} for this system is
            defined to be $\Phi(t,s)=\Psi(t)\Psi(s)^{-1}$. Calculate it.
            \textbf{Note:} The reason why we call this a ``transition
            matrix'' is that the effect of multiplication by the
            matrix $\Phi(t,t_0)$ is to transition the homogeneous
            solution forward in time from $t_0$ to $t$.
        \item Armed with the transition matrix corresponding to the
            homogeneous ODE, we can now calculate the solution to the
            inhomogeneous ODE using the variation of parameters formula
            \[
                \bf{x}(t)=\Phi(t,0)\bf{x}(0)
                +\int_{0}^{t}\Phi(t,s)\bf{F}(s)~ds.
            \]
            Use this formula to calculate the solution to the IVP.
    \end{enumerate}
    \textbf{Hint:} Throughout this problem, you should use the formula
    \[
        \m{a&b\\c&d}^{-1}=\f{1}{ad-bc}\m{d&-b\\-c&a}
    \]
    to calculate the inverse of a $2\times2$ matrix. (Unfortunately,
    there are no simple formulas for larger matrices.)
\end{problem}
\begin{problem}[3 cont.]
    \textbf{Note:} To learn more about fundamental matrices and the
    derivation of the variation of parameters formula, read Section
    7.9 of Boyce and diPrima.
\end{problem}
\begin{solution}
    \null\vfill
\end{solution}
\end{document}
