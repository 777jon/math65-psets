\documentclass[12pt,letterpaper]{hmcpset}
\usepackage[margin=1in]{geometry}
\usepackage{graphicx}
\usepackage{hyperref}
\usepackage{enumitem}

% info for header block in upper right hand corner
\name{ }
\class{Math 65}
\assignment{HW 12}
\duedate{Thursday, June 2, 2016}

\newcommand{\f}[2]{\frac{#1}{#2}}
\newcommand{\pn}[1]{\left(#1\right)}
\newcommand{\abs}[1]{\left|#1\right|}
\newcommand{\bk}[1]{\left|ft[#1\right]}
\renewcommand{\t}[1]{\text{#1}}
\renewcommand{\bf}[1]{\mathbf{#1}}
\renewcommand{\labelenumi}{{(\alph{enumi})}}

\begin{document}
\problemlist{1, 2, 3, 4, 5}

\begin{problem}[1]
    Consider the nonlinear system of DEs
    \begin{align*}
        x'&=x-y+x^2-xy\\
        y'&=-y+x^2
    \end{align*}
    \begin{enumerate}
        \item Find the nullclines of this system. Use the nullclines
            to perform ``sign analysis.'' In other words, determine the
            signs of $x'$ and $y'$ in each of the regions of the phase
            plane delineated by the nullclines.
        \item Locate the equilibrium points of this system of
            differential equations.
        \item For each equilibrium point,
        \begin{enumerate}[label=(\roman*)]
            \item write down the linearized system of DEs about that
                point,
            \item calculate the eigenvalues of the linearized system,
            \item use the eigenvalues to determine the behavior
                (unstable/stable, node/spiral, etc.) of the linearized
                system about that point,
            \item if possible use that information about the
                linearized system to infer something about the nature of
                the equilibrium point of the original nonlinear system
                of DEs.
        \end{enumerate}
        \item Use the information about these equilibrium points and
            the sign analysis to sketch orbits by hand on your phase
            plane.
        \item Use mathematical software to plot a phase plane portrait
            of the system and compare it to your hand sketch.  Please do
            this only after you've tried your best to do what you can by
            hand.  Feel free to use the computer-generated picture to
            correct any details that you might have missed.
    \end{enumerate}
\end{problem}
\newpage
\begin{solution}
    \null\vfill
\end{solution}
\newpage

\begin{problem}[2]
    The nonlinear stability of an equilibrium point cannot be reliably
    predicted if all that is known is that the linearized system has a
    center there.  This can be demonstrated by these two systems of
    equations.
    \begin{alignat*}{3}
        \dot{x}&=y+x(x^2+y^2)&\qquad\qquad\raisebox{-.1in}[0mm][0mm]
        {\t{and}}\qquad\qquad\dot{x}&=y-x(x^2+y^2)\\
        \dot{y}&=-x+y(x^2+y^2)&\dot{y}&=-x-y(x^2+y^2)
    \end{alignat*}
    \begin{enumerate}
        \item Linearize both systems about the origin (the only
            equilibrium point). What type of equilibrium point does the
            linearized system have at the origin?
        \item Use computer software to show what the solution
            trajectories actually look like if you choose initial
            conditions that are close to the origin.  Based on these
            calculations, what is the true nature of the equilibrium
            point at the origin for each system of equations?
        \item (Extra credit) Convert both systems of equations from
            Cartesian coordinates $(x,y)$ to polar coordinates
            $(r,\theta)$, where $r^2=x^2+y^2$, etc.  What do the
            converted equations tell you about the stability of the
            origin in each case?
    \end{enumerate}
\end{problem}
\begin{solution}
    \vfill
\end{solution}
\newpage

\begin{problem}[3]
    On the last assignment, you began analyzing the competitive
    species population model:
    \begin{align*}
        x'(t)&=x(r_1-a_1x-b_1y)\\
        y'(t)&=y(r_2-a_2y-b_2x)
    \end{align*}
    All of the parameters in the DE above ($r_1$, $a_1$, $b_1$, $r_2$,
    $a_2$, $b_2$) are positive, and we restrict our attention to
    $x\geq0$ and $y\geq0$ only.  Depending on these six constants,
    there are four different (nondegenerate) cases to be considered.
    \begin{center}
        \begin{tabular}{c|c|c}
            Case & Condition & Behavior\\\hline
            1& 
            $\f{r_1}{a_1}<\f{r_2}{b_2}$ \&
            $\f{r_1}{b_1}<\f{r_2}{a_2}$
            &
            $x$ goes extinct for any positive initial population
            values\\
            2& 
            $\f{r_1}{a_1}>\f{r_2}{b_2}$ \&
            $\f{r_1}{b_1}>\f{r_2}{a_2}$
            &
            $y$ goes extinct for any positive initial population
            values\\
            3& 
            $\f{r_1}{a_1}<\f{r_2}{b_2}$ \&
            $\f{r_1}{b_1}>\f{r_2}{a_2}$
            &
            \begin{tabular}{c}
                both $x,y$ approach coexistence for any positive
                initial\\
                population values
            \end{tabular}\\
            4& 
            $\f{r_1}{a_1}>\f{r_2}{b_2}$ \&
            $\f{r_1}{b_1}<\f{r_2}{a_2}$
            &
            \begin{tabular}{c}
                one species will go extinct, depending on initial\\
                population values
            \end{tabular}
        \end{tabular}
    \end{center}
    Please refer to the solutions for Homework \#11 for complete
    details.
    \begin{enumerate}
        \item In all four cases, there are three extinction
            equilibrium points: $(0,0)$, $(r_1/a_1,0)$, and
            $(0,r_2/a_2)$. Use local stability analysis to determine the
            nature of each equilibrium point in all four cases.
        \item In cases 3 and 4, there is an additional coexistence
            equilibrium point at
            \[
                \bf{x}_{\t{eq}}=\pn{x_{\t{eq}},y_{\t{eq}}}
                =\pn{\f{r_1a_2-b_1r_2}{a_1a_2-b_1b_2},
                \f{a_1r_1-r_2b_2}{a_1a_2-b_1b_2}}
            \]
            that is viable.  Pick some values for the six constants so
            that you are in case 3.  Use those values to perform local
            stability analysis to determine the nature of this
            equilibrium point.  Repeat for case 4. (We are using
            specific values for the constants to capture the behavior
            in cases 3 and 4 since the algebra is a little more
            intense if you don't do that.)
        \item Synthesize all of the calculations that you've performed
            to explain why the populations behave the way they do in all
            four cases.
        \item (Extra credit) Perform local stability analysis for
            arbitrary values of the six constants in cases 3 and 4.
            \textbf{Hint:} The algebra is not as bad as you think if you
            are strategic about it.
    \end{enumerate}
\end{problem}
\newpage
\begin{solution}
    \null\vfill
\end{solution}
\newpage

\begin{problem}[4]
    The \textit{Brusselator} system of equations,
    \begin{align*}
        \dot{x}&=a+x^2y-(1+b)x\\
        \dot{y}&=bx-x^2y,
    \end{align*}
    is a mathematical model of a certain chemical reaction.  (Google
    ``Brusselator'' and ``Belousov Zhabotinsky reaction'' to find out
    more.)  The parameters $a$ and $b$ are related to reaction rates,
    and are therefore positive numbers.  These equations have the
    property that certain choices of parameters lead to long-term
    oscillatory solutions.
    \begin{enumerate}
        \item Where is the only equilibrium point of this system?
        \item Assuming that $a=1$, use local stability analysis to
            describe how the stability of this equilibrium point depends
            on $b$.  In particular, what happens when $b=2$?  Remember
            not to draw any conclusions about the nonlinear differential
            equation system that are not warranted by the results of
            local stability analysis.
        \item (Extra credit) Use a computer to generate some phase portraits
            to verify your work and to investigate what happens.  You will see
            an attracting ``limit cycle'' (long-term oscillatory solution)
            appear around the equilibrium point as $b$ increases past 2.
    \end{enumerate}
\end{problem}
\begin{solution}
    \vfill
\end{solution}
\newpage

\begin{problem}[5]
    Last June, the Deep Space Climate Observatory, which performs
    real-time solar wind monitoring (important for warning about
    geomagnetic storms that can disrupt power grids,
    telecommunications, etc.) reached its final destination at ``L1,"
    nearly a million miles away from the earth.  L1 is one of five
    equilibrium points, called Lagrange points, that exist in a
    three-body system consisting of two large bodies (such as the sun
    and the earth) and one small body (such as a satellite).\\\\
    In this problem you will determine the stability of L1 in a
    simplified case where we ignore the fact that the large bodies are
    orbiting each other.  Consider the sun (mass $M_S$) and earth
    (mass $M_E$) to be a fixed distance $d$ apart. Let $x(t)$ be the
    distance of the satellite (with mass $m$) from the sun on the line
    joining the sun and earth.
    \begin{enumerate}
        \item Using Newton's Second Law, write the governing
            differential equation for the motion of the satellite along
            this line. (Look up the equation for the gravitational force
            between two objects if you need to.)  You may assume that
            the satellite is positioned in between the sun and
            earth. Write this second-order ODE as a system of first
            order equations.  Find the equilibrium position of the
            satellite that is between the sun and earth.
        \item Linearize your differential equations about the
            equilibrium point and assess its stability using local
            stability analysis.
    \end{enumerate}
    \textbf{Note \#1:} You don't have to look up the values of the
    masses of the sun and earth in this problem. You can keep them as
    variables, along with the gravitational constant. Don't be afraid
    of a little algebra. Be strategic about how you simplify things.\\
    \textbf{Note \#2:} Google ``Lagrange points'' if you'd like to
    know more or if you'd like to see the mathematics required if the
    rotating frame of reference is taken into account. You can read
    more about the DSCO at \url{http://www.nesdis.noaa.gov/DSCOVR/}.
\end{problem}
\begin{solution}
    \vfill
\end{solution}
\end{document}
