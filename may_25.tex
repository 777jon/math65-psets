\documentclass[12pt,letterpaper]{hmcpset}
\usepackage[margin=1in]{geometry}
\usepackage{graphicx}
\usepackage{url}

% info for header block in upper right hand corner
\name{ }
\class{Math 65}
\assignment{HW 7}
\duedate{Wednesday, May 25, 2016}

\newcommand{\pn}[1]{\left( #1 \right)}
\newcommand{\abs}[1]{\left| #1 \right|}
\newcommand{\bk}[1]{\left[ #1 \right]}

\newcommand{\RR}{\mathbb{R}}
\newcommand{\m}[1]{\begin{bmatrix} #1 \end{bmatrix}}

\renewcommand{\labelenumi}{{(\alph{enumi})}}

\begin{document}

\problemlist{1, 2, 3, 4, 5, 6}

\begin{problem}[1]
    For each system of DEs shown below, explain whether it is
    \begin{itemize}
        \item linear or nonlinear
        \item homogeneous (undriven) or inhomogeneous (driven)
        \item autonomous or nonautonomous.
    \end{itemize}
    Also, for any linear system of DEs, rewrite the system using
    vector \& matrix notation.
    \begin{enumerate}
        \item $\begin{cases}
            x'&\hspace{-.1in}=\sin(t)x+e^{ty}y+3t^2\\
            y'&\hspace{-.1in}=\cos(t)x+e^{-ty}y
            \end{cases}$
        \item $\begin{cases}
            x'&\hspace{-.1in}=3x+4xy\\
            y'&\hspace{-.1in}=2x-3xy
            \end{cases}$
        \item $\begin{cases}
            x'&\hspace{-.1in}=3tx+4ty\\
            y'&\hspace{-.1in}=6t^2x+\sin(t)y
            \end{cases}$
        \item $\begin{cases}
            x'&\hspace{-.1in}=3x+4y+\sqrt{t}\\
            y'&\hspace{-.1in}=\mbox{\hspace{4.5mm}}-3y-\sqrt{t}
            \end{cases}$
        \item $\begin{cases}
            x'&\hspace{-.1in}=3x+2y\\
            y'&\hspace{-.1in}=-x-y
            \end{cases}$
    \end{enumerate}
\end{problem}
\newpage
\begin{solution}
    \null\vfill
\end{solution}
\newpage

\begin{problem}[2]
    Here is a general $n$th-order ODE for $y(x)$.
    \[
        a_n(x)\frac{d^ny}{dx^n}+a_{n-1}(x)\frac{d^{n-1}y}{dx^{n-1}}
        +\cdots+a_1(x)\frac{dy}{dx}+a_0(x)y(x)=0
    \]
    Write it as a system of first-order ODEs, in matrix form.
\end{problem}
\begin{solution}
    \vfill
\end{solution}
\newpage

\begin{problem}[3]
    Consider the second-order ODE $y"(t)+2y'(t)+2y(t)=0$.
    \begin{enumerate}
        \item First, solve this ODE using techniques that you learned
            in Math 45. Express the general solution in two forms: (1)
            complex exponentials and (2) sines and cosines (with no
            complex numbers).
        \item Next, convert this second-order ODE to an equivalent
            first-order DE system. Find the general solution to this
            system of equations. Use Euler's Identity to rearrange
            things so that you get a real-valued solution in the
            end. You should find that your work agrees with your answer
            from part (a).
    \end{enumerate}
\end{problem}
\begin{solution}
    \vfill
\end{solution}
\newpage

\begin{problem}[4]
    Solve the initial-value problem $\mathbf{x}'(t)=A\mathbf{x}(t)$
    with
    \[
        A=\m{5&5&2\\-6&-6&-5\\6&6&5}
        \quad\text{and}\quad
        \mathbf{x}(0)=\m{1\\-1\\1}
    \]
    Express your answer as a real-valued function.
\end{problem}
\begin{solution}
    \vfill
\end{solution}
\newpage

\begin{problem}[5]
    At $t=0$ (i.~e.~noon) a student takes a fast-dissolving
    antihistamine capsule. The antihistamine is absorbed from the GI
    tract (stomach and intestines) into the blood system and then
    excreted. Let $x_1$ be the amount of antihistamine in the GI tract
    and $x_2$ be the amount in the blood system. Assume that the rate
    of absorption from the GI tract into the blood system is $k_1x_1$
    and the rate of excretion from the bloodstream (via the kidneys)
    is $k_2x_2$, corresponding to the following compartment diagram.
    Also, assume that $k_2<k_1$ (the rate of excretion is faster than
    the rate of absorbtion).
    \begin{center}
        \includegraphics[scale=0.7]{img/may_24_5}
    \end{center}
    \begin{enumerate}
        \item Explain why the amount of antihistamine in the body
            satisfies the system
            \begin{align*}
                \frac{dx_1}{dt}&=-k_1x_1\\
                \frac{dx_2}{dt}&=k_1x_1-k_2x_2
            \end{align*}
            together with the initial conditions $x_1(0)=\alpha$ and
            $x_2(0)=0$ where $\alpha$ is the initial amount of
            antihistamine in the GI tract just after the capsule has
            dissolved.
        \item This system of equations is a \textit{cascading} system
            of equations in that the first equation only involves $x_1$
            and the second equation involves both $x_1$ and
            $x_2$. Therefore, you can solve the first equation by
            itself, then plug in the solution for $x_1$ into the second
            equation and solve for $x_2$. Solve the system in this
            fashion.
        \item Next solve the system of equations again using linear
            algebra (eigenvalues and eigenvectors of the system matrix).
        \item When does the amount of antihistamine in the blood
            system reach a maximum? What is the maximum amount? Your
            answers will be in terms of $\alpha$, $k_1$, and $k_2$
            (\textbf{Hint:} Your final answer for the maximum amount can
            be written quite simply.)
    \end{enumerate}
\end{problem}
\newpage
\begin{solution}
    \null\vfill
\end{solution}
\newpage

\begin{problem}[6]
    Make up an initial-value problem involving a system of linear,
    first-order differential equations that has the property that its
    solution exists only for $a<t<b$, where $a$ and $b$ are numbers of
    your choosing.  Use ODEToolkit (\url{http://odetoolkit.hmc.edu})
    to draw the solution trajectories. Make sure you label your axes
    and show that the solution only exists for $a<b<t$.
\end{problem}
\begin{solution}
    \vfill
\end{solution}
\end{document}
